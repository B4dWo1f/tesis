\chapter{Graphene Bilayer}
\section{Introduction}
When $N$ vacancies are introduced, $N$ in-gap states appear. If we consider only one orbital per site, then all the states appear at $E=0$ as dictated by the Lieb's theorem (bipartite lattice at half filling in the limit of $U\to 0$).

When an electric field is applied to graphene bilayer (GBL) a gap is opened at the Dirac points. For reasonable electric fields, the gap opens linearly with the electric field, until at some point it saturates. The maximum experimentally observed value of the gap is\cite{Zhang2009} $\Delta=250meV$.

We are going to consider a 0D system, a hexagonal island with armchair edges. It contains over 90000 atoms (91812, to be precise) and the bilayer is stalked in a Bernal way.
The electric field is introduced as an energy imbalance between the two graphene layers, and the vacancies are considered as an infinite on-site energy. For now we will just consider that the vacancies are placed along the $X$ axis (namely, $\alpha=0$).

Since we consider a 0D system, no bands can be defined, hence only the spectrum can be calculated. Since the full diagonalization of such a system would be almost impossible (for an average computer) we will take advantage of the Lanczos diagonalization method\cite{Arnoldi1951,Lanczos1950}, already available in a number of standard libraries.
This computation method provides the $n$ eigenstates closest to a given energy, in our case we will consider a few more eigenstates than in-gap states should appear in order to be sure that the conduction and valence states are also captured.

For a given electric field, the result of our calculation is just the spectrum: a set of eigenenergies and eigenvectors. When we repeat the calculation for different electric fields we can see the smooth variation of the states as a function of the electric field.

\section{Two vacancies}
The typical evolution of the spectrum of an island with 2 vacancies is shown in Fig.~\ref{2vacs}. For a given
