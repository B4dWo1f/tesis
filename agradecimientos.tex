\chapter*{Agradecimientos}
Joaquín, Jose, Spinograph, Belén %TODO


Gracias a Jose entre otras muchas cosas por sus eternas palabras de sabiduría: "Piénsalo en la base diagonal" y "haz primero el caso simple, y luego lo complicas".\\

Gracias a toda la gente de Spinograph, Christian, Denis, Luis, Jose (el malo), Pep, Jose (el bueno), French guy, Mario, Mehrdad, Sowmya, Luis, pero especialmente a Francesca, flat-mate por unos días y hermana en esta cruzada llamada doctorado, a Wenjing, que hizo de mi mes en San Sebastian una estancia de lo más confortable, y por ultimo a Regina, que con sus conversaciones en San Sebastián y en Madrid me ha ayudado más de lo que posiblemente se imagine. Me alegro mucho mucho de que Spinograph nos haya dado la oportunidad de conocernos.

Por supuesto también tengo que hacer hueco para agradecer a la magnífica gente de Graphenea, que me acogieron con los brazos abiertos y me hicieron pasar un gran mes, especialmente Oihana e Iker, pero sin olvidar a Illargi, Mireia, Marta, Alba y Amaia.\\


Mirando hacia atrás en este camino de la física no puedo evitar recordar aquella primera clase de Rafa Casero con su integral triple el primer día y las caras aterradas de muchos estudiantes... ese terror sirvió de cemento para crear un gran grupo de muones a quien hay que agradecer discusiones, apoyo y moral durante toda la carrera.
Las largas jornadas en la biblioteca (y los largos descansos en la biblioteca) fueron la gasolina que me permitió terminar la carrera medio-cuerdo. Blanca, Carlos F. Carlos CR, Bu, Jezú, Jorge, Marina, Raquel, Rubén, Rubo, Victor, Zuli (así como los muones honoríficos, Mar, Marina M...), una gran parte de esta tesis se debe a vosotros. Muchas gracias.\\

Me gustaría hacer una mención especial a Carlos CR ya que ciertas épocas turbulentas nos han acercado más y más, y he encontrando siempre un apoyo en él. Además de ser un físico brillante y un gran programador (cuando abandone definitivamente Mathematica), es un gran amigo y le estoy muy agradecido por su apoyo y su cariño.\\

También se merece una mención especial Víctor, con quien he recorrido muchos metros verticales, porque sólo con él son posibles esos magníficos planes...\\
Mainz-Braga-Madrid-Contreras(Valencia)-Madrid-Braga en un finde? Parece factible...\\
Prepárate, que en cuanto vuelva a estar en forma Yosemite se nos va a quedar pequeño.\\


Por supuesto tengo que agradecer la cálida acogida de la gente de Alicante durante el tiempo que estuvimos allí. Desde el primer día Jesús, Taner, Marta, Maria José... nos hicieron sentir uno más del departamento, pero hay dos personas que se merecen una mención especial. A Miguel querría agradecer las conversaciones sobre la vida y los metros de roca compartidos. A Bernat, a parte de las noches New Orleandinas, querría agradecerle su apoyo constante y su comprensión cuando la deslocalización me superaba. A pesar de todo cada vez que volvía a Alicante Bernat me ha hecho sentir que volvía a casa.\\

No puede faltar en esta lista los compañeros de faena de Braga, Tareq, Noelia, Diogo y Jose (otra vez) que con las discusiones de barcos y aceites han hecho esta etapa mucho más llevadera y divertida.\\


Y es obligatorio en esta tesis dar las gracias de una manera especial a Ester que ha estado siempre a un Telegram-azo de distancia, disponible 24/7 para la duda más tonta, mi inseguridad más estúpida o el vacío existencial más profundo, sin importar las circunstancias. En esta tesis me ha ayudado con figuras y con atascos mentales, con días de aburrimiento y días de stress infinito, con apoyo y con hostias con la palma abierta, según requiriera la ocasión.
Ha estado dispuesta a ayudarme desde que tengo memoria y en cualquier situación, soportando todas mis imbecilidades (que no han sido pocas) y mis grandes metidas de pata (que también ha habido unas cuantas).
Literalmente no tengo palabras para expresar todo lo que te debo, pero estoy completamente seguro de que sin tu apoyo y tu cariño no habría sobrevivido ni el colegio, ni la universidad, ni el doctorado...ni la vida.\\



Por último, querría darle las gracias a Bu por muchas cosas, por toda la ayuda que me prestó durante la carrera, por enseñarme una gran parte de lo q sé, sobre física y sobre mi mismo. Por su sonrisa perenne y su mirada siempre positiva. Por su falta de rencor y su simplicidad.






Porque fuimos nosotros contra el mundo y por algunos años ganamos...\\
Porque fuimos el mejor equipo... de verdad que lo fuimos...
