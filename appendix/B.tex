\chapter{Units in the hyperfine coupling}
The magnetic moment for a spin $1/2$ particle is
\begin{equation}
\vec{m} = g\mu\frac{1}{2}\vec{\sigma}
\end{equation}
as expected, $\vec{m}$ has units of $\si{\eV/\tesla}$ since $g$ is dimensionless, and $\mu$ is
\begin{equation}
  \mu = \frac{q\hbar}{2m} \longrightarrow
  [\mu] = \si{\coulomb\eV\s\per\kilogram} = \si{\eV\per\tesla}
  % [\mu] = \frac{C\cdot eV\cdot s}{kg} = \frac{eV}{T}
\end{equation}

The hyperfine contact term:
\begin{equation}
  H_{hf} = -\mu_0\frac{2}{3} \vec{m}_e\vec{m}_p\delta(\vec{R})
\end{equation}


Each of the magnetic moments have units of $\left[m_i\right] = \si{\eV\per\tesla}$ and the vacuum magnetic permeability has units of $\left[\mu_0\right]=\si{\newton/\ampere\squared}$, and the Dirac's delta has units $[\delta(\vec{R})]=1/R^3$

So the units are correct\footnote{we do not care here about the factors}:
% \begin{equation}
%   \underbrace{\frac{N}{A^2}}_{\mu_0}
%   \underbrace{\frac{eV}{T}}_{\vec{m}_e}
%   \underbrace{\frac{eV}{T}}_{\vec{m}_p}
%   \underbrace{\frac{1}{m^3}}_{\delta(\vec{R})}
% \end{equation}
\begin{equation}
  \si{\newton\per\ampere\squared}\si{\eV\per\tesla}\si{\eV\per\tesla}\si{\per\m\cubed}
  % \frac{N}{A^2}\frac{eV}{T}\frac{eV}{T}\frac{1}{m^3}
\label{units}
\end{equation}
It is useful to keep in mind some relations among units:
\begin{equation}
  \si{\tesla} \sim
  \si{\newton\per\ampere\per\m}\sim\si{\volt\s\per\m\squared} \sim
  \si{\kilogram\per\coulomb\per\s}  \quad;\quad
  \si{\V}\sim\si{\eV\per\coulomb}  \quad;\quad
  \si{\ampere}\sim\si{\coulomb\per\s}
\end{equation}
Now continuing the equation \eqref{units}
\begin{equation}
  \si{\newton\per\ampere\squared}\frac{\si{\eV}}{\si{\newton\per\ampere\per\m}}\si{\eV\per\tesla}\si{\per\m\cubed} \sim
  \si{\per\ampere}\si{\eV\squared\per\tesla}\si{\per\m\squared} \sim
  \si{\s\per\coulomb}\si{\eV\squared\per\tesla}\si{\per\m\squared} \sim
  \si{\V\s\per\m\squared}\si{\eV\per\tesla}\sim \si{\eV}
\end{equation}

So the hyperfine coupling has units of energy as expected. Now, this coupling is usually provided in $MHz$. For instance in Kane's paper:
\begin{equation}
  \frac{2\mathcal{A}}{h} = \SI{58}{\MHz} = \frac{4\pi\mathcal{A}}{\hbar}
\end{equation}

Finally!! the discrepancy with the Cohen book that says:
\begin{equation}
  \frac{\mathcal{A}\hbar}{2\pi} = \SI{1420}{\MHz}
\end{equation}
comes from the fact that their spin operators $\vec{I}$ and $\vec{S}$ have eigenvalues $\pm\hbar/2$, so the hyperfine coupling $\mathcal{A}$ should have units of $\si{\per\eV\per\s\squared}$ so in the end:
\begin{equation}
  \mathcal{A}\vec{I}\vec{S} = \si{\per\eV\per\s\squared}\cdot\si{\eV\s}\cdot \si{\eV\s} = \si{\eV}
\end{equation}
