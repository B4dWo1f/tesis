\chapter{Quantum operations}
\label{operations}
Quantum gates are usually represented as matrices. A gate which acts on k qubits is represented by a $2^k\times2^k$ unitary matrix.
The number of qubits in the input and output of the gate have to be equal. The action of the gate on a specific quantum state is found by multiplying the vector which represents the state by the matrix representing the gate. In the following, the vector representation of a single qubit is:
\begin{equation}
  v_0\ket{\up} + v_1\ket{\down} \rightarrow \left[\begin{array}{c}
  v_0 \\
  v_1
  \end{array}\right]
\end{equation}

and the vector representation of two qubits is
\begin{equation}
  v_{00}\ket{\up\up} + v_{01}\ket{\up\down} + v_{10}\ket{\down\up} + v_{11}\ket{\down\down} \rightarrow
  \left[\begin{array}{c}
  v_{00} \\
  v_{01} \\
  v_{10} \\
  v_{11}
  \end{array}\right]
\end{equation}
where $\ket{ab}$ is the basis vector representing a state where the first qubit is in the state $\ket{a}$ and the second qubit in the state $\ket{b}$, usually states as this are also expressed as direct product: $\ket{ab}=\ket{a}\otimes\ket{b}$.

When dealing with 1 Qubit operations the used basis will be:
\begin{equation}
  \mathcal{B} = \left\{\ket{\up},\ket{\down}\right\}
\end{equation}

For 2 qubit operations the basis will be:
\begin{equation}  %XXX Check order
  \mathcal{B} = \left\{\ket{\uaw,\Uaw},\ket{\uaw,\Daw},
                       \ket{\daw,\Uaw},\ket{\daw,\Daw}\right\}
\label{basis2}
\end{equation}


\section{Single Qubit Operations}

\subsection{Hadamard gate}
The Hadamard gate acts on a single qubit. It maps:
\begin{equation}
  \ket{\uaw} \longrightarrow \frac{\ket{\uaw}+\ket{\daw}}{\sqrt{2}} \quad\quad;\quad\quad
  \ket{\daw} \longrightarrow \frac{\ket{\uaw}-\ket{\daw}}{\sqrt{2}}
\end{equation}
which means that a measurement will have equal probabilities to become 1 or 0.
It represents a rotation of $\pi$ about the axis $(\hat{x}+\hat{z})/ 2$.
Equivalently, it is the combination of two rotations, $\pi$ about the X-axis followed by $\pi /2$ about the Y-axis. It is represented by the Hadamard matrix:
\begin{equation}
  H=\frac{1}{\sqrt{2}}\left(\begin{array}{cc}
  1 & 1 \\
  1 & -1
  \end{array}\right)
\end{equation}
$H$ is an unitary matrix since $HH^\dagger=\mathbb{I}$ where $\mathbb{I}$ is the identity matrix.

\subsection{Pauli matrices gates}
Unsurprisingly this family of operators are represented by the Pauli matrices:
\begin{equation}
  X = \left(\begin{array}{cc}
  0 & 1 \\
  1 & 0
  \end{array}\right)\quad;\quad
  Y = \left(\begin{array}{cc}
  0 & -i \\
  i & 0
  \end{array}\right)\quad;\quad
  Z = \left(\begin{array}{cc}
  1 & 0 \\
  0 & -1
  \end{array}\right)
\end{equation}
Notice that these matrices are also unitary $XX^\dagger=YY^\dagger=ZZ^\dagger= \mathbb{I}$ where $\mathbb{I}$ is the identity matrix.

The Pauli-X gate is the quantum equivalent of a NOT gate (with respect to the standard basis of eigenvectors of $S_z$: $\mathcal{B}=\left\{\ket{\uaw},\ket{\daw}\right\}$, and it equates to a rotation of the Bloch sphere around the X-axis by $\pi$ radians. Summarizing, it maps:
\begin{equation}
  \ket{\uaw} \longrightarrow \ket{\downarrow} \quad\quad;\quad\quad
  \ket{\downarrow} \longrightarrow \ket{\uaw}
\end{equation}
Due to this nature, it is sometimes called bit-flip.

The Pauli-Y gate corresponds to a rotation around the $Y$ axis of the Bloch sphere by $\pi$ radians. It maps the states:
\begin{equation}
  \ket{\uaw} \longrightarrow i\ket{\downarrow} \quad\quad;\quad\quad
  \ket{\downarrow} \longrightarrow -i\ket{\uaw}
\end{equation}

Analogously the Pauli-Z gate describes a rotation around the Z-axis of the Bloch sphere by $\pi$ radians. Thus, it is a special case of a phase shift gate (explained later) with $\theta=\pi$. It leaves the basis state $\ket{\uaw}$ unchanged but modifies the basis state $\ket{\downarrow}$:
\begin{equation}
  \ket{\uaw} \longrightarrow \ket{\uaw} \quad\quad;\quad\quad
  \ket{\downarrow} \longrightarrow -\ket{\downarrow}
\end{equation}
Because of this behavior, it is sometimes called phase-flip.

\subsection{$\sqrt{\text{NOT}}$ gate}
This gate is called like this because $\sqrt{\text{NOT}}\sqrt{\text{NOT}}=\text{NOT}=X$. It is represented by the following matrix:
\begin{equation}
  \sqrt{\text{NOT}} = \frac{1}{2}\left(\begin{array}{cc}
  1+i & 1-i \\
  1-i & 1+i
  \end{array}\right)
\end{equation}
This gate is also unitary: $\sqrt{\text{NOT}}\sqrt{\text{NOT}}^{\dagger}=\mathbb{I}$

\subsection{Phase-Shift gates}
This is a family of single-qubit gates that leave the basis state $\ket{\uaw}$ unchanged but modifies the state $\ket{\downarrow}$:
\begin{equation}
  \ket{\uaw} \longrightarrow \ket{\uaw} \quad\quad;\quad\quad
  \ket{\downarrow} \longrightarrow e^{i\phi}\ket{\downarrow}
\end{equation}
The probability of measuring a $\ket{\uaw}$ or $\ket{\downarrow}$ is unchanged after applying this gate, however it modifies the phase of the quantum state. This is equivalent to tracing a horizontal circle (a line of latitude) on the Bloch sphere by $\phi$ radians.
\begin{equation}
  R_{\phi} = \left(\begin{array}{cc}
  1 & 0 \\
  0 & e^{i\phi}
  \end{array}\right)
\end{equation}
For the special case of $\phi = \pi$ this gate is the same as the Pauli-Z gate.

\section{Two Qubits operations}
We will use simple arrows ($\uaw$, $\downarrow$) to designate the first spin and double arrows ($\Uaw$, $\Daw$) to designate the second one.
For describing the state of both qubits we will use the basis \eqref{basis2}


\subsection{Swap gate}
The swap gate, as expected, swaps the states two qubits. It is represented by the matrix:
\begin{equation}
  \text{SWAP} = \left(\begin{array}{cccc}
  1 & 0 & 0 & 0 \\
  0 & 0 & 1 & 0 \\
  0 & 1 & 0 & 0 \\
  0 & 0 & 0 & 1
  \end{array}\right)
\end{equation}
Basically it maps the basis states as follows:
\begin{equation}
  \begin{split}
    \ket{\uaw\Uaw} \longrightarrow \ket{\uaw\Uaw}\\
    \ket{\uaw\Daw} \longrightarrow \ket{\downarrow\Uaw}\\
    \ket{\downarrow\Uaw} \longrightarrow \ket{\uaw\Daw}\\
    \ket{\downarrow\Daw} \longrightarrow \ket{\downarrow\Daw}
  \end{split}
\end{equation}



\subsection{$\sqrt{\text{SWAP}}$ gate}
This is an important gate since it is universal in the sense that any quantum many-qubit gate can be constructed from only $\sqrt{\text{SWAP}}$ and single-qubit gates. As expected $\sqrt{\text{SWAP}}\sqrt{\text{SWAP}}=\text{SWAP}$. Hence, the matrix representing this gate is:
\begin{equation}
  \sqrt{\text{SWAP}} = \left(\begin{array}{cccc}
  1 & 0 & 0 & 0 \\
  0 & \frac{1}{2}(1+i) & \frac{1}{2}(1-i) & 0 \\
  0 & \frac{1}{2}(1-i) & \frac{1}{2}(1+i) & 0 \\
  0 & 0 & 0 & 1
  \end{array}\right)
\end{equation}

\section{Controlled gates}
Controlled gates act on 2 or more qubits, where one or more qubits act as a control for some operation.
\subsection{CNOT gate}
The controlled NOT gate (or CNOT) acts on 2 qubits, and performs the NOT operation on the second qubit only when the first qubit is $\ket{\downarrow}$, and otherwise it leaves it unchanged. It is represented by the matrix
\begin{equation}
  \text{CNOT} = \left(\begin{array}{cccc}
  1 & 0 & 0 & 0 \\
  0 & 1 & 0 & 0 \\
  0 & 0 & 0 & 1 \\
  0 & 0 & 1 & 0
  \end{array}\right)
\end{equation}

\subsection{General controlled gate}
More generally if $U$ is a gate that operates on single qubits with matrix representation
\begin{equation}
  U = \left(\begin{array}{cc}
  u_{00} & u_{01} \\
  u_{10} & u_{11}
  \end{array}\right)
\end{equation}
We define the controlled-U gate as a gate that operates on the second qubit only when the state of the first qubit is $\ket{\downarrow}$. This allows the use of one of the qubis as control. It maps the basis states as follows.
\begin{equation}
  \begin{split}
    &\ket{00}\rightarrow\ket{00}\\
    &\ket{01}\rightarrow\ket{01}\\
    &\ket{10}\rightarrow\ket{1}\otimes U\ket{0}=\ket{1}\otimes \left(u_{00}\ket{0}+u_{10}\ket{1}\right)\\
    &\ket{11}\rightarrow\ket{1}\otimes U\ket{1}=\ket{1}\otimes \left(u_{01}\ket{0}+u_{11}\ket{1}\right)
  \end{split}
\end{equation}
The matrix representing the controlled U is
\begin{equation}
  \text{C-U} = \left(\begin{array}{cccc}
  1 & 0 & 0 & 0 \\
  0 & 1 & 0 & 0 \\
  0 & 0 & u_{00} & u_{01} \\
  0 & 0 & u_{10} & u_{11}
  \end{array}\right)
\end{equation}

For the specific case where $U=X$ we would recover the CNOT gate.


\section{Other relevant gates}
For the proposes of quantum computation, the 3-qubit gates are also important, for the sake of completeness  we review the two main gates of this kind. When dealing with 3-qubit states we will expand our basis as follows:
\begin{equation}
  \mathcal{B}'=\left\{\ket{\uaw\Uaw\tilde{\Uaw}},
                     \ket{\uaw\Uaw\tilde{\Daw}},
                     \ket{\uaw\Daw\tilde{\Uaw}},
                     \ket{\uaw\Daw\tilde{\Daw}},
                     \ket{\downarrow\Uaw\tilde{\Uaw}},
                     \ket{\downarrow\Uaw\tilde{\Daw}},
                     \ket{\downarrow\Daw\tilde{\Uaw}},
                     \ket{\downarrow\Daw\tilde{\Daw}}
              \right\}
\end{equation}
For the sake of simplicity we will use $0$ ($1$) to denote the up (down) state, and its position in the ket to specify the qubit, with this notation the basis would be:
\begin{equation}
  \mathcal{B}=\left\{\ket{000},\ket{001},\ket{010},\ket{011},
                     \ket{100},\ket{101},\ket{110},\ket{111}\right\}
\end{equation}

\subsection{Toffoli gate}
The Toffoli gate is a 3-qubit controlled gate that could be defined as a CCNOT gate. This gate is also an universal gate for classical computation.

If the first two bits are in the state $\ket{\downarrow}$, it applies a Pauli-X on the third bit, else it does nothing. Since it is the quantum analog of a classical gate, it is completely specified by its truth table.
\begin{table}[h!]
\begin{center}
\begin{tabular}{rrr|rrr}
% \hline
\multicolumn{ 3}{c|}{Input} & \multicolumn{ 3}{c}{Output} \\ \hline
0 & 0 & 0 & 0 & 0 & 0 \\ %\hline
0 & 0 & 1 & 0 & 0 & 1 \\ %\hline
0 & 1 & 0 & 0 & 1 & 0 \\ %\hline
0 & 1 & 1 & 0 & 1 & 1 \\ %\hline
1 & 0 & 0 & 1 & 0 & 0 \\ %\hline
1 & 0 & 1 & 1 & 0 & 1 \\ %\hline
1 & 1 & 0 & 1 & 1 & 1 \\ %\hline
1 & 1 & 1 & 1 & 1 & 0 \\ %\hline
\end{tabular}
\end{center}
\vspace{-15pt}\caption{Truth table for the Toffoli gate}
\end{table}
Its matrix representation is as follows
\begin{equation}
  \text{Toffoli} = \left(\begin{array}{cccccccc}
  1 & 0 & 0 & 0 & 0 & 0 & 0 & 0 \\
  0 & 1 & 0 & 0 & 0 & 0 & 0 & 0 \\
  0 & 0 & 1 & 0 & 0 & 0 & 0 & 0 \\
  0 & 0 & 0 & 1 & 0 & 0 & 0 & 0 \\
  0 & 0 & 0 & 0 & 1 & 0 & 0 & 0 \\
  0 & 0 & 0 & 0 & 0 & 1 & 0 & 0 \\
  0 & 0 & 0 & 0 & 0 & 0 & 0 & 1 \\
  0 & 0 & 0 & 0 & 0 & 0 & 1 & 0
  \end{array}\right)
\end{equation}

It can be also described as the gate which maps:
\begin{equation}
  \ket{a,b,c}\longrightarrow\ket{a,b,c\otimes ab}
\end{equation}
Here the $\otimes$ symbol denotes the NOT operator, hence it means that the $c$ bit is NOTed only if $a$ and $b$ are in the state 1

 \subsection{Fredkin gate}
 The Fredkin gate is also known as the CSWAP gate, this is, a controlled swap gate. It is also universal for classical computation. As with the Toffoli gate it has the useful property that the numbers of 0s and 1s are conserved throughout.
 \begin{table}[h!]
 \begin{center}
 \begin{tabular}{rrr|rrr}
 % \hline
 \multicolumn{ 3}{c|}{Input} & \multicolumn{ 3}{c}{Output} \\ \hline
 $C$ & $I_1$ & $I_2$ & $C$ & $O_1$ & $O_2$ \\ \hline
 0 & 0 & 0 & 0 & 0 & 0 \\ %\hline
 0 & 0 & 1 & 0 & 0 & 1 \\ %\hline
 0 & 1 & 0 & 0 & 1 & 0 \\ %\hline
 0 & 1 & 1 & 0 & 1 & 1 \\ %\hline
 1 & 0 & 0 & 1 & 0 & 0 \\ %\hline
 1 & 0 & 1 & 1 & 1 & 0 \\ %\hline
 1 & 1 & 0 & 1 & 0 & 1 \\ %\hline
 1 & 1 & 1 & 1 & 1 & 1 \\ %\hline
 \end{tabular}
 \end{center}
 \vspace{-15pt}\caption{Truth table for the Fredkin gate}
 \end{table}
 Its matrix representation is as follows
 \begin{equation}
   \text{Fredkin} = \left(\begin{array}{cccccccc}
   1 & 0 & 0 & 0 & 0 & 0 & 0 & 0 \\
   0 & 1 & 0 & 0 & 0 & 0 & 0 & 0 \\
   0 & 0 & 1 & 0 & 0 & 0 & 0 & 0 \\
   0 & 0 & 0 & 1 & 0 & 0 & 0 & 0 \\
   0 & 0 & 0 & 0 & 1 & 0 & 0 & 0 \\
   0 & 0 & 0 & 0 & 0 & 0 & 1 & 0 \\
   0 & 0 & 0 & 0 & 0 & 1 & 0 & 0 \\
   0 & 0 & 0 & 0 & 0 & 0 & 0 & 1
   \end{array}\right)
 \end{equation}

\section{Universal quantum gates}
% Informally, a set of universal quantum gates is any set of gates to which any operation possible on a quantum computer can be reduced, that is, any other unitary operation can be expressed as a finite sequence of gates from the set. Technically, this is impossible since the number of possible quantum gates is uncountable, whereas the number of finite sequences from a finite set is countable. To solve this problem, we only require that any quantum operation can be approximated by a sequence of gates from this finite set. Moreover, for unitaries on a constant number of qubits, the Solovay-Kitaev theorem guarantees that this can be done efficiently.
%
One simple set of two-qubit universal quantum gates is the Hadamard gate ($H$), the$\pi/8$ gate $R(\pi/4)$, and the controlled NOT gate. This means that we can perform any possible operation on a quantum computer by performing a finite sequence of these gates.

This is not strictly true, but close enough (Solovay-Kitaev theorem) \red{UNDERSTAND}
%
% A single-gate set of universal quantum gates can also be formulated using the three-qubit Deutsch gate $D(\theta)$, which performs the transformation
% \begin{equation}
%   \ket{a,b,c}\rightarrow\left\{\begin{array}{ll}
%   icos(\theta)\ket{a,b,c}+sin(\theta)\ket{a,b,1-c} & \quad\text{for} a=b=1\\
%   \ket{a,b,c}  & \quad\text{otherwhise}
%   \end{array}\right.
% \end{equation}
%
% The universal classical logic gate, the Toffoli gate, is reducible to the Deutsch gate, $D(\frac{\pi}{2})$, thus showing that all classical logic operations can be performed on a universal quantum computer.
