%~~~~~~~~ Chapter ~~~~~~~~
\chapter{Introduction}

%
% Industrial revolution, PC/internet revolution, Quantum revolution
% short-sighted expectations
% quantum technologies
% graphene bubble
%
%

We are today on the verge of a technological revolution, a quantum leap\footnote{pun intended} that holds the promise for world-changing advances. 
The development of Quantum Technologies has barely started and it already promises a revolution in many different fields, from medicine and pharmacology, computation, encryption, any application in which optimization is a problem and even the understanding of quantum mechanics itself.
There is probably nothing riskier than trying to make technological predictions, but even governments around the world are taking action to be in the forefront of this revolution\cite{QTF} and big companies are investing huge amounts of money in these topics. But let us start from the beginning.
\medbreak

Starting at the end of the XIX century, a number of phenomena pointed the attention of physicists towards what we call today quantum physics. In the beginning, these phenomena just had in common the quantization of some physical quantities.




The world has seen all sorts of revolutions all throughout history. While not necessarily every revolution has turned the world into a better place, when we account just for the scientific or technological revolutions, this is usually the case.
\medbreak

%At the end of the XVIII century, the invention of the steam engine and its later introduction in the industrial processes allowed a huge step forward for many societies and, for better or worse\cite{}%climate
%, reshaped the world.
%The Industrial Revolution changed the way we travel and work, it changed the scale at which humans interacted with the planet and, arguably, the universe.
%The whole world has seen how virtually any index measuring quality of life has been raising ever since\cite{}.\footnote{Sadly this trend has not been homogeneous at all around the globe and these changes have been much more clear in some countries than others.}

%Similarly, another huge revolution came along with
The worldwide introduction of computers in most households and the construction of a worldwide web interconnecting ``everyone'' in the world has been one of the biggest revolutions over the last 30 years. The information technologies have changed the job-market landscape, the way humans learn, shop, the way we socialize even the way we fight wars. Even in their first stages computers were able to put two men on the moon and bring them back with less computational power than the cheapest smartphone available today.
Our daily habits, our leisure time, even crucial historical events\cite{Alhindi2012} have been determined by the use of computers and the internet.

For this revolution to happen, many agents had to be in place. The need for secure communication channels during World War II pushed the development of information theory and the later Cold War pushed scientists and engineers to build computers. But the bare necessity is not enough to develop the necessary technology. The first computers used \emph{macroscopic transistors} and \emph{macroscopic bits} when they were introduced. At first these essential pieces were handmade and manufactured one by one. It was not until the research in semiconductor physics developed microscopic versions of these pieces that the computers became cheap and easy to produce.
\medbreak

\newpage
For nowadays computers to exist, it was necessary the development of Quantum Mechanics. Starting with the beginning of the XX century, physicist realised that Physics was not completely understood but rather there were many mysteries when 



Naturally these revolutions did not just happen, it took many years of hard work and research to stablish the basis for these technologies to emerge. In particular, for computers to be created and be popularized it was necessary a deep understanding of the laws of quantum physics which led to the mastery of the semiconductor technologies.



\newpage

%%%%%%%%%%%%%%%%%%%%%%%%%%%%%%%%%%%%%%%%%%%%%%%%%%%%%%%%%%%%%%%%%%%%%%%%%%%%%%%%
The relation between research in fundamental Physics and technological innovations is not always easy to see. In the best of cases, it usually takes some years, at least a few decades, for fundamental research to find its way into the industry.
While this is a completely fair question, it is quite short-sighted. Let us think of the latest revolution that changed our society.
The introduction of computers in our everyday life and the access to a global network of communications was a revolution, maybe even comparable to the industrial revolution.

We have had computers in most households for the last 30 years, broad access to the internet for over 20, yet, trying to pin-point the fundamental research that took us there is quite hard, if not impossible.
We can think of the first transistor\footnote{actually there is a previous patent, from 1930,\cite{diode} but no hard proof that it was ever built or used.},
the building block of every electronic device, which would take us to the 40's\cite{Ross}, but it would be unimaginable to reach that point without at least some understanding of the physics of semiconductors which had been studied for the previous 50 years. It would be insane trying to understand semiconductors without, at least, a few notions of quantum mechanics, electromagnetism...
We could line up all the most brilliant minds of humanity that set the path towards today's technology: Landau, Bardeen, Bloch Dirac, Heisenberg, Bragg, Brillouin, Faraday, Gauss... countless others... Ask them about the application of their research, and not one of them would be able to describe a personal computer or the internet.

That is why asking the researchers about the usefulness of their research is like asking newborn babies what will be their job position in 42 years time.

Of course we have our own vision of what the research may bring upon the world, but let us keep in mind that this vision is nothing but a childish dream, since time and time again the applications have exceeded the wildest expectations of the researchers.\\


At the end of the XIX century it was believed that Physics was solved. Today we know it is not. In particular, Condensed Matter Physics is an area in which even when we know the physical laws governing every interaction there is still much that we do not understand. Quantum Mechanics gives us the tools to describe any problem that we want to describe. From superconductors to topological phases or spin liquids, the Hamiltonian is known and yet we cannot solve it. The last fifty or sixty years of condensed matter physicist have been devoted to approximations and simplifications that would allow us to strip the problems to its basic components in an attempt to understand the microscopical mechanism behind so many interesting phenomena.
The thorough investigation of a variety of problems while not completely successful in every area, has had a great side effect. The cumulative experience of decades of condensed matter physics research has granted us the ability to control matter with atomic precision in a wide range of situations. The fulfillment of such a dream\cite{Feynman1982}

Quantum technologies\cite{QTF}

The research on semiconductor physics can be hard to pin-point, 


In this work we study a variety of properties regarding defected graphene bilayer. Starting from the basic concepts of graphene, we build our way to the role of sp$^3$-defects in graphene bilayer such as chemisorbed Hydrogen adatoms.
First we study the properties of these defects one by one in isolated scenarios. Then we wonder the effects of neighboring defects
